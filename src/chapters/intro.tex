\chapter{Introduction}
\label{chapter:intro}

Designing and building a conversational agent that impersonates a historical personality, a public figure or a know celebrity and can answer user input questions about that person regarding the life and work of that person is a challenging task implying various methods like information retrieval, natural language processing (NLP) and question answering (QA). These methods are broad sub-fields and integral parts of the domain of Artificial Intelligence (AI).

These sub-fields and the difficulties accompanying them are described, in short, in the next three paragraphs.

{\em Information Retrieval.} Information retrieval is a method of gathering relevant information pertaining to a given problem from a collection of resources. The problems with this approach stretch from knowing how people use and process information and how the knowledge can be represented, to the text indexing methods and relevantly scoring parts of the information extracted from such texts.

{\em Natural Language Processing.} Natural language processing is a field of AI that explores how the natural language can be analyzed, understood and manipulated by a computer. The research in NLP attempts to find out how humans express and understand language so that appropriate tools and techniques can be developed to make computers perform the desired tasks \cite{Chowdhury2003}. NLP is an essential part of creating a conversational agent that interacts "naturally" and "coherently" with a human through language. The main difficulties that arise in NLP are of semantic nature: understanding of the context in which an analyzed sentence appears and overcoming the ambiguities that appear in natural human language. The problem pertaining to the conversational agent in this thesis is that it is needed to understand the human language from both the biographical corpus exctracted for a given personality and the question posed by the user.

{\em Question Answering Systems.} QA systems are programs that attempt to find the most relevant and accurate answer for a given question. The main issue that appears in QA is the need for taxonomies depending on the set of conceptual categories that covers the wanted domain for the potential questions. Other issues are: question rephrasing, answer extraction and answer formulation \cite{Burger2001}.

All the aforementioned difficulties add up to make the task of creating a conversational agent challenging.

The attempts of programming machines to interact with humans via natural language and "behave" as if they are human started when Alan Turing raised the question "Can machines think?" \cite{Turing1950} more than 60 years ago. Since then, a considerable part of the computer science community has dedicated its time and resources to create computer programs that can convince people that the answer to this question is "Yes".

Over the years, notable advancements have been made towards this goal. The testament to that are various programs, from simple agents, like ELIZA \cite{Weizenbaum1966}, A.L.I.C.E. \cite{Wallace2009} or Freudbot \cite{Heller2005}, to more complex and "smarter" agents, like Cleverbot\footnote{\href{http://www.cleverbot.com/}{Cleverbot Webpage}, \url{http://www.cleverbot.com/}} and IBM Watson \cite{Ferrucci2010}. The mentioned agents are summarily described in the \nameref{chapter:related-work} chapter.

\begin{comment}
\section{Natural Language Processing}
\label{sec:nlp}

Natural Language Processing (NLP)

\section{Conversational Agent}
\label{sec:intro-conversational-agent}

\section{Question Answering}
\label{sec:intro-qa}

Question Answering (QA)
\end{comment}

% HELPERS

% reference to \labelindexref{Section}{sub-sec:proj-objectives}.
% \abbrev{CS}{Computer Science}
% \labelindexref{Figure}{img:report-framework}.
%\fig[scale=0.5]{src/img/reporting-framework.pdf}{img:report-framework}{Reporting Framework}
% citations \cite{iso-odf}.
% \index{Ultimate answer to all knowledge}


\begin{comment}
We can also have tables... like \labelindexref{Table}{table:reports}.

\begin{center}
\begin{table}[htb]
  \caption{Generated reports - associated Makefile targets and scripts}
  \begin{tabular}{l*{6}{c}r}
    Generated report & Makefile target & Script \\
    \hline
    Full Test Specification & full_spec & generate_all_spec.py  \\
    Test Report & test_report & generate_report.py  \\
    Requirements Coverage & requirements_coverage &
    generate_requirements_coverage.py   \\
    API Coverage & api_coverage & generate_api_coverage.py  \\
  \end{tabular}
  \label{table:reports}
\end{table}
\end{center}
\end{comment}
